\newpage
\section{Netzwerkanalyse}

\begin{multicols}{2}
\subsection{Serieschaltung von Widerständen}
$ R_{ges} = \sum\limits_{k=1}^{n}{R_k} $ \\

\subsection{Parallelschaltung von Widerständen}
$ \frac{1}{R_{ges}} = \sum\limits_{k=1}^{n}\frac{1}{R_k}\ \ \ R_{ges2} = \frac{R_1 \cdot R_2}{R_1 + R_2} $ \\

\subsection{Spannungsteiler}
\includegraphics[width=0.15\textwidth]{pics/spannungsteiler}\\
$ U_1 = U \cdot \frac{R_1}{R_1 + R_2}$ \\
$ U_2 = U \cdot \frac{R_2}{R_1 + R_2}$ \\

\end{multicols}

\subsection{Kirchhof}
\begin{multicols}{4}
\includegraphics[width=0.15 \textwidth]{pics/kirchhof/masche}\\
Maschensatz: Summe aller Spannungen =0.\\
	$U_1+U_2+U_3-10V=0$
	$10V-U_4+U_q-U_5=0$
	$U_1+U_2+U_3-U_4+U_q-U_5=0$\\
\includegraphics[width=0.25\textwidth]{pics/kirchhof/knoten}\\
Konotensatz: Summer aller Str"ome = 0, zufliessende Ströme positiv abfliessende Ströme negativ.\\
$4A-8A+5A-2A+I=0A$\\
\end{multicols}

\subsection{Stern Dreieck Umwandlung}
\begin{multicols}{3}
\includegraphics[width=0.3\textwidth]{pics/stern-dreieck}
\subsubsection{$ \triangle \rightarrow $ Y}
$ R_0 = R_1 + R_2 + R_3 $ \\
$ R_a = \frac{R_1R_2}{R_0} = \frac{R_1R_2}{R_1 + R_2 + R_3} $ \\
$ R_b = \frac{R_2R_3}{R_0} = \frac{R_2R_3}{R_1 + R_2 + R_3} $ \\
$ R_c = \frac{R_1R_3}{R_0} = \frac{R_1R_3}{R_1 + R_2 + R_3} $ \\
\subsubsection{Y $ \rightarrow \triangle $}
$ G_0 = G_a + G_b + G_c = \frac{1}{R_a} + \frac{1}{R_b} + \frac{1}{R_c} $ \\
$ R_1 = R_aR_cG_0 = R_aR_c(\frac{1}{R_a} + \frac{1}{R_b} + \frac{1}{R_c}) $ \\
$ R_2 = R_aR_bG_0 = R_aR_b(\frac{1}{R_a} + \frac{1}{R_b} + \frac{1}{R_c}) $ \\
$ R_3 = R_bR_cG_0 = R_bR_c(\frac{1}{R_a} + \frac{1}{R_b} + \frac{1}{R_c}) $ \\
\end{multicols}

\subsection{Begriffe, Definition}
\begin{tabular}{lll}
Knoten & (Verbindungspunkte) & Anzahl: k \\
Zweige & (Verbindungen zwischen Knoten) & Anzahl: z \\
Vollständiger Baum & Nicht geschlossener Linienzug, verbindet sämtliche Knoten& \\
Baumzweige ("Aste) & &  Anzahl: k-1\\
Verbindungszweige (Sehnen) & Alle Zweige, die nicht zum Baum gehören & Anzahl: z-(k-1)=z-k+1\\
Kreis & Geschlossene Folge von Zweigen, enthält keinen Knoten zweimal & \\
Masche & Kreis, der entweder innerhalb oder ausserhalb keine Zweige enthält & \\
Planares Netzwerk & Netzwerk ohne Kreuzung von Zweigen & \\
\end{tabular}

\subsection{Wahl von Kreisen}
\begin{itemize}
\item Beim Einzeichnen eines kreises einen Zweig davon markieren. Diesen Zweig bei den weiteren Kreisen nicht mehr verwenden.
\item Man zeichnet einen vollständigen Baum. Jeder Kreis soll eine andere Sehne (und nur diese) enthalten $\rightarrow$ "`Basiskreis"'
\item Man wählt als Kreise lauter Maschen (z.B. alle inneren). Voraussetzung: Netzwerk planar
\end{itemize}
\newpage
\subsection{Maschen- und Kreisstrommethode}
Der Unterschied zwischen diesen beiden Analysemethoden ist einfach, dass die Kreisstrommethode allgemeiner ist. Bei der Maschenmethode wird jede Masche als Kreis behandelt, bei der Kreisstrommethode können Kreise auch als "`Nicht-Maschen"' angesehen werden. Besonders bei der Maschenstrommethode kann die Matrix durch Einzeichngen der Bezugspfeile in dieselbe Richtung enorm vereinfacht werden.
\begin{figure}[ht]
  \begin{minipage}[t]{8 cm} %BASTEL!!
	  \centering
	  Maschenstrommethode:\\
	  \includegraphics[width=8cm]{pics/dcnet/maschenmethode}
	 \end{minipage}
	 \begin{minipage}[t]{11 cm}
	  \centering
	  Kreisstrommethode:\\
	  \includegraphics[width=11cm]{pics/dcnet/kreisstrommethode} 
  \end{minipage}
\end{figure}

Passende Matrix zur Maschenstrommethode:

$\left[\begin{array}{ccc}
	+R_1+R_2+R_3 &\hervor{-}R_2 &\hervor{-}R_3  \\
	\hervor{-}R_2 &+R_2+R_4+R_5
	&\hervor{-}R_5\\ 
	\hervor{-}R_3 &\hervor{-}R_5 &+R_3+R_5+R_6
	\end{array}\right] \bullet 
	\left[\begin{array}{l}j_1\\j_4\\j_6\end{array}\right] = 
	\left[\begin{array}{c}\hervor{+}U_{q1}\\0\\
	\hervor{-}U_{q6}
\end{array}\right]$\\

Passende Matrix zur Kreisstrommethode:

$\left[\begin{array}{ccc}
	+R_1+R_2+R_3 &\hervor{+}R_2 &\hervor{+}R_2+R_3\\
	\hervor{+}R_2 &+R_2+R_4+R_5
	&\hervor{+}R_2+R_4\\ 
	\hervor{+}R_2+R_3 &\hervor{+}R_2+R_4 &+R_2+R_3+R_4+R_6
	\end{array}\right] \bullet 
	\left[\begin{array}{l}j_1\\j_2\\j_3\end{array}\right] = 
	\left[\begin{array}{c}\hervor{+}U_{q1}\\0\\
	\hervor{+}U_{q6}
\end{array}\right]$

Der Vorteil der Kreisstrommethode gegenüber der Maschenstrommethode ist der,dass ein gewünschter Strom direkt aus der Gleichung ausgerechnet werden kann. In diesem Beispiel wäre bspw. $I_5 = j_2$.

Die Ausrechnung einer Matrix mit dem TI-89 funktioniert so (Beispiel  Maschenmethode):
$$rref([R_1+R_2+R_3\hervor{,}-R_2\hervor{,}-R_3,U_{q1}\hervor{;}
-R_2\hervor{,} \ldots])$$

\subsection{Knotenpotentialmethode}
\begin{figure}[ht]
  \begin{minipage}[lt]{7 cm}
    \includegraphics[width=7cm]{pics/dcnet/knotenpotentialmethode} 
  \end{minipage}
  \begin{minipage}[rt]{9.35 cm} %BASTEL!!
  Bei dieser Methode wird immer mit \textit{Leitwerten} $G$ und (idealen) \textit{Stromquellen}
gerechnet. Siehe auch S. 33 im Formelbuch.
  \end{minipage}
\end{figure}

$\left[\begin{array}{ccc}
+G_1+G_2+G_4 &\hervor{-}G_2 &\hervor{-}G_4\\
\hervor{-}G_2 &+G_2+G_3+G_5
&\hervor{-}G_5\\ 
\hervor{-}G_4 &\hervor{-}G_5 &+G_4+G_5+G_6
\end{array}\right] \bullet 
\left[\begin{array}{l}U_1\\U_2\\U_3\end{array}\right] = 
\left[\begin{array}{c}\hervor{+}I_{q1}\\0\\
\hervor{-}I_{q6}\end{array}\right]$

\subsection{Behandlung idealer Quellen}
1. \underline{Kreisstrom- bzw. Maschenstrommethode:}
\begin{itemize}
	\item Spannungsquellen (lineare oder ideale) können ohne weiteres in die Netzwerkgleichungen einbezogen werden. Die Quellenspannung erscheint "`rechts"' des Gleichheitszeichens als sogenannte Störfunktion.
	\item \underline{Lineare Stromquellen} ($G_{i} \not = 0$) werden in \underline{Spannungsquellen} umgewandelt
	\item Ideale Stromquellen erforden eine Sonderbehandlung --> Quellenströme gleich als Kreisströme benutzen(Quellen in Sehnen von Basiskreisen legen!)--> \underline{Pro ideale Stromquelle vermindert sich die Zahl der Netzwerkgleichungen, aber auch die Zahl der Unbekannten um eins!}
\end{itemize}
2. \underline{Knotenpotentialmethode:}
\begin{itemize}
	\item Stromquellen(lineare oder ideale) können problemlos in die Netzwerkgleichungen einbezogen werden.
	\item \underline{Lineare Spannungsquellen}($R_{i} \not = 0$)werden in \underline{Stromquellen} umgewandelt
	\item Ideale Spannungsquellen erfordern eine Sonderbehandlung --> eine Klemme der Spannungsquelle musss am Bezugsknoten liegen\\
	Sind mehrere ideale Spannungsquellen vorhanden, und haben diese \underline{keinen} gemeinsamen Knoten, so kann das eben beschriebene verfahren nur bei einer Quelle angewandt werden. --> andere Quellen verschieben
\end{itemize}

\subsection{Strom- Spannungsquellen-Verschiebung}
\begin{multicols}{2}
\begin{center}
\includegraphics[width=0.35\textwidth]{pics/dcnet/UQuellenver}\\
\end{center}
\begin{center}
\includegraphics[width=0.5\textwidth]{pics/dcnet/IQuellenver}\\
\end{center}
\end{multicols}

\subsection{Aspekte zur Wahl der Methode}
\begin{tabular}{lll}
	\multicolumn{3}{l}{Die \textit{Art der gegebenen Quellen} ist wohl die wichtigste Entscheidungsgrundlage:}\\
	&Kreisstrommethode: &Spannungsquellen\\
	&Knotenpotentialmethode: &Stromquellen\\
	\multicolumn{3}{l}{Die \textit{Anzahl Gleichungen} kann auch entscheidend sein:}\\
	&Kreisstrommethode: &$z-k+1-$ Anzahl idealer Stromquellen\\
	&Knotenpotentialmethode: &$k-1-$ Anzahl idealer Spannungsquellen\\
	\multicolumn{3}{l}{Ein kleines Argument dürften die \textit{gesuchten Grössen} sein (Umformung durch $U=R \cdot I$):}\\
	&Kreisstrommethode: &(Sehnen-) Ströme\\
	&Knotenpotentialmethode: &(Knoten-) Spannungen
\end{tabular}