\section{Schaltungen mit Quellen}

\begin{multicols}{2}
\textbf{Ideale Spannungsquelle mit parallellem Widerstand}
\columnbreak
\textbf{Ideale Stromquelle mit seriellem Widerstand}
\end{multicols}

\begin{multicols}{4}
\begin{circuitikz}
	\draw
	(2, 0) -- (0, 0) 
	to[american voltage source] ++(0,2) -- ++(2,0)
	to [open, o-o] ++(0,-2) -- (1, 0)
	to [european resistor] ++(0, 2) ;
\end{circuitikz}

\columnbreak

%Spannungsquelle alleine
\begin{circuitikz}
	\draw
	(1.5, 0) -- (0, 0) 
	to[american voltage source] ++(0,2) -- ++(1.5,0)
	to [open, o-o] ++(0,-2) -- (1, 0);
%	to [european resistor] ++(0, 2) ;
\end{circuitikz}

\columnbreak


\begin{circuitikz}
	\draw
		(2, 0) -- (0, 0) 
		 to[american current source] ++(0, 1.6)
		 to[european resistor] ++(2, 0)
		 to [open, o-o] ++(0,-1.6) ;
\end{circuitikz}

\columnbreak

% Stromquelle alleine
\begin{circuitikz}
	\draw
	(1.5, 0) -- (0, 0) 
	to[american current source] ++(0, 2) -- ++(1.5, 0)
	to [open, o-o] ++(0,-2) ;
	
\end{circuitikz}
\end{multicols}


\begin{multicols}{2}
	\textbf{Ideale Stromquelle parallel zu idealer Spannungsquelle}
	\columnbreak
	\textbf{Ideale Stromquelle in Serie zu idealer Spannungsquelle}
\end{multicols}



\begin{multicols}{4}
	\begin{circuitikz}
		\draw
		(2, 0) -- (0, 0) 
		to[american current source] ++(0,2) -- ++(2,0)
		to [open, o-o] ++(0,-2) -- (1, 0)
		to [american voltage source] ++(0, 2) ;
	\end{circuitikz}
	
	\columnbreak
	
	%Spannungsquelle alleine
	\begin{circuitikz}
		\draw
		(1.5, 0) -- (0, 0) 
		to[american voltage source] ++(0,2) -- ++(1.5,0)
		to [open, o-o] ++(0,-2) -- (1, 0);
		%	to [european resistor] ++(0, 2) ;
	\end{circuitikz}
	\columnbreak	
	\begin{circuitikz}
		\draw
		(2, 0) -- (0, 0) 
		to[american current source] ++(0, 1.6)
		to[american voltage source] ++(2, 0)
		to [open, o-o] ++(0,-1.6) ;
	\end{circuitikz}
	\columnbreak
	
	% Stromquelle alleine
	\begin{circuitikz}
		\draw
		(1.5, 0) -- (0, 0) 
		to[american current source] ++(0, 2) -- ++(1.5, 0)
		to [open, o-o] ++(0,-2) ;
		
	\end{circuitikz}
	
\end{multicols}

\begin{multicols}{2}
	\textbf{Mehrere Id. Stromquellen und Widerstände parallel}
	\columnbreak
	\textbf{Mehrere Id. Spannungsquellen und Widerstände in Serie}
\end{multicols}

\begin{minipage}{6cm}
\begin{circuitikz}
	\draw
	(2, 0) -- (0, 0) 
	to[american current source] ++(0,2) -- ++(1,0)
	to [european resistor] ++(0, -2) -- ++ (1, 0)
	to [american current source] ++(0, 2) -- ++(-1, 0) -- ++(2, 0)
	to [european resistor] ++(0, -2) -- ++ (-1, 0) -- ++ (3, 0) -- ++(-1, 0) 
	to [american current source] ++(0,2) -- ++(-1, 0) -- ++(2, 0)
	to [open, o-o] ++(0, -2);
\end{circuitikz}
\end{minipage}
\begin{minipage}{3.5cm}
\begin{circuitikz}
	\draw
	(2, 0) -- (0, 0) 
	to[american current source] ++(0,2) -- ++(2,0)
	to [open, o-o] ++(0,-2) -- (1, 0)
	to [european resistor] ++(0, 2) ;
\end{circuitikz}
\end{minipage}
\begin{minipage}{6cm}
\begin{circuitikz}
	\draw
	(5, 0) -- (0, 0) 
	to[american voltage source] ++(0, 1.5)
	to[european resistor] ++(2, 0)
	to[american voltage source] ++(1, 0)
	to[european resistor] ++(2, 0)
	to [open, o-o] ++(0,-1.5) ;
\end{circuitikz}
\end{minipage}
\begin{minipage}{2.5cm}
\begin{circuitikz}
	\draw
	(2, 0) -- (0, 0) 
	to[american voltage source] ++(0, 1.5)
	to[european resistor] ++(2, 0)
	to [open, o-o] ++(0,-1.5) ;
\end{circuitikz}
\end{minipage}

