\section{Felder}
\subsection{Das elektrische Feld, Elektrostatik}
Elektrostatik: Lehre von den ruhenden elektrischen Ladungen und deren Wirkung auf ihre Umgebung, insbesondere auf andere Ladungen\\
\subsubsection{Was ist ein Feld? Eigenschften, Darstellung}
\textbf{Feld}:Zustand des Raumes, bei dem jedem Punkt innerhalb eines bestimmten Gebietes eine physikalische Gr�sse - die Feldgr�sse - zugeordnet ist.\\
Ein Feld ist eine 2D Fl�che mit der L�nge l und der Breite b, wo der Bauer seine Kartoffeln anpflanzt.\\
Die Feldgr�sse, z.B. Temperatur oder Geschwindigkeit, ist i.a vom Ort im Raum abh�ngig. Man spricht von einer Ortsfunktion oder Punktfunktion(wie in der Mathematik).\\
\textbf{Skalarfeld}: Die Feldgr�sse ist ein Skalar, d.h. eine ungerichtete Gr�sse, die durch eine Zahl(und eine Masseinheit) vollst�ndig bestimmt ist. Beispiel: Temperatur\\

\textbf{Vektorfeld}: Die Feldgr�sse ist ein Vektor. Ein Vektor legt Betrag und Richtung fest Beispiele: Windgeschwindigkeit, Kraft\\
\newpage
\subsubsection{Darstellung von Vektorfeldern}
\begin{multicols}{2}
\begin{itemize}
	\item In einzelnen Punkten des Raumes bzw. der Ebene wird der zugeh�rigen \underline{Feldvektoren als Pfeil} eingezeichnet.
	\item Durch \underline{Feldlinien}: Das sind gerichtete Linien, deren Tangenten in jedem ihrer Punkte dieselbe Richtung haben wie der Feldvektor.\\
\end{itemize}
\includegraphics[width=0.25\textwidth]{pics/felder/Feldvektoren}
\end{multicols}
\subsubsection{Einige spezielle Vektorfelder}
\begin{itemize}
\begin{multicols}{2}
	\item Ein Vektorfeld heisst \textbf{homogen}, falls jedem Punkt seines Definitionsbereichs derselbe Vektor zugeordnet ist. Der Feldvektor ist also im ganzen Gebiet nach Betrag und Richtung konstant (unabh�ngig vom Ort). Die Feldlinien sind parallele Geraden bzw. Strecken.	
	\begin{itemize}
		\item Elektrisches FEld zwischen zwei parallelen Kondensatorplatten in gen�gend grosser Entfernung vom Rand\\
		\item Geschwindigkeitsfeld eines rein translatorisch bewegten K�rpers\\
		\includegraphics[width=0.2\textwidth]{pics/felder/PlattenkondensatorFeld.png}

	\end{itemize}
	\item Ein Vektorfeld heisst \textbf{eben}, falls ein rechtwinkliges Koordinatensystem so gew�hlt werden kann, dass alle Feldvektoren zur x-y-Ebene parallel sind und nicht von der z-Koordinate abh�ngen. Die Feldlinienbilder in allen Ebenen senkrecht zur z-Achse sind also identisch
	\begin{itemize}
		\item Elektrisches Feld eines Koaxialkabels (Zylindrisches Koaxialfeld)\\
		\item Elektrostatisches Feld zwischen langen, parallelen Leitern\\
		\item Magnetfeld zweier (oderer mehrerer) paralleler, gerader Leiter, die von konstanten Str�men durchflossen werden\\
	\end{itemize}
\end{multicols}
	\begin{multicols}{2}
	\item In der \underline{Elektrostatik} beginnen die Feldlinien des elektrischen Feldes auf positiven Ladungen und enden auf negativen.\\
	Die positiven Ladungen k�nnen als Quellen, die negativen als Senken des elektrischen Feldes aufgefasst werden. Man spricht von einem (reinen) \textbf{Quellenfeld}.\\
		\includegraphics[width=0.5\textwidth]{pics/felder/Quellenfeld}\\
	\item Im Gegensatz zum reinen Quellenfeld steht das \textbf{Wirbelfeld}. Hier sind die Feldlinien in sich geschlossen.\\
	\includegraphics[width=0.2\textwidth]{pics/felder/Wirbelfeld_1}\includegraphics[width=0.2\textwidth]{pics/felder/Wirbelfeld_RotierendeScheibe}
	\end{multicols}
\end{itemize}
