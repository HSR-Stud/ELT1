\section{Einleitung}
\subsection{Die elektrische Ladung}

\begin{itemize}
	\item Ladungen im Atommodell von Bohr-Sommerfeld\\
	\begin{itemize}
			\item Kern 
			$\begin{cases}
				Protonen:  & \text{- tragen je eine positive Elementarladung(+e)}\\ & \text{- Masse: $1.7 \cdot 10^{-27}$kg,gegen 2000x schwerer als 					
				Elektronen}\\
				Neutronen: & \text{- elektrisch neutral}\\ & \text{- Masse ungefähr wie Proton}
			\end{cases}$\\
			\item Hülle Elektronen
			\begin{itemize}
				\item[-] tragen je eine negative Elementarladungen (-e)\\
				\item[-] kleine Masse ($9.1 \cdot 10^{-31}kg$)\\
				\item[-] umkreisen den Kern aus Kreis- oder Ellipsenbahnen\\
			\end{itemize}
		\end{itemize}
	\item Masseinheit der Ladung Q: $ \lbrack Q \rbrack = 1 Coulomb = 1 C = 1 As$, $ e = 1.602\cdot 10^{-19} C$, $1 C = 6.25\cdot 10^{18}$ Elektronen
	\item Aufnahme von Elektronen $\rightarrow$ neg. Ladung überwiegt $\rightarrow$ Anion\\
		Abgabe von Elektronen $\rightarrow$ pos. Ladung überwiegt $\rightarrow$ Kation\\
	
\end{itemize}

\subsection{Der elektrische Strom}
Stromrichtung $\rightarrow$ Bewegungsrichtung der positiven Ladungsträger\\
\begin{multicols}{2}
	\subsubsection{Stromstärke I}
	$$ I=\frac{\Delta Q}{\Delta t} \hspace{10pt}  \lbrack I \rbrack = A $$\\
	\subsubsection{Stromdichte J}
	$$ J=\frac {I}{A}  \hspace{10pt}  \lbrack J \rbrack = \frac {A}{m^2} $$\\
\end{multicols}

\subsubsection{Geschwindigkeit der freien Elektronen}
\begin{itemize}
	\item Ungeordnete thermische Bewegung, ähnlich wie bei Molekülen eines Gases:ca. 100 km/s (bei Raumtemperatur)\\
	\item Driftgeschwindigkeit = mittlere Fortbewegungsgeschwindigkeit der Elektronen in  (Gegen-)Stromrichtung: max. einige mm/s.\\
	\item Impulsgeschwindigkeit = Geschwindigkeit des Signals, das die Elektronen zur Drift veranlasst: Lichtgeschwindigkeit $\approx$ 300000 km/s\\
\end{itemize}
%\newpage

\subsection{Die elektrische Spannung}

\begin{multicols}{2}
	$$E=\frac{F}{Q} \qquad \text{Feldstärke }\lbrack E \rbrack = \frac{V}{m}$$\\
	$$U_{AB} = \frac{W_{AB}}{Q} = E \cdot l_{AB} \qquad
	\lbrack U \rbrack = \frac{\lbrack W \rbrack}{\lbrack Q \rbrack} = \frac{J}{C} = V$$\\
\end{multicols}
	
\subsubsection{Das elektrische Potential $ \varphi $}
\underline{Definition}: Eine Spannung gegenüber einem (gemeinsamen) Bezugspunkt bezeichnet man auch als elektrisches Potential $\varphi$.\\
Dem Bezugspunkt wird meistens (aber nicht zwangsläufig) das Potential $\varphi = 0$ zugeordnet.\\
 Die Spannung $U_{AB}$ zwischen zwei Punkten A und B (mit den Potentialen $\varphi _{A} $ bzw. $\varphi _{B}$) erhält man als Potentialdifferenz:\\ $ U_{AB}= \varphi _{A} - \varphi _{B}$\\

\subsection{Leistung und Energie}
	$$ P = \frac{\Delta W}{\Delta t}= I \cdot U = \frac{\Delta Q \cdot U}{\Delta t} \qquad \lbrack P \rbrack = W (Watt)$$\\
	$$ \Delta W = U \cdot I \cdot \Delta t = \underbrace {e \cdot U_{AK} = \frac{1}{2} \cdot m_{e} \cdot v^2}_\text{Elektronenröhre} \qquad \lbrack W \rbrack = Ws = Nm = J (Joul)$$
	