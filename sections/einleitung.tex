\newpage
\section{Einleitung}
\subsection{Grössen}


\subsection{Die elektrische Ladung}
Alle physikalischen Erscheinungen, die wir als elektrisch bezeichnen, haben ihre Ursache in
ruhenden oder bewegten elektrischen Ladungen. Jede elektr. Ladung ist ein ganzzahliges Vielfaches von der Elementarladung e. 
\linebreak

\begin{itemize}
	\item Ladungen im Atommodell von Bohr-Sommerfeld\\
	\begin{itemize}
			\item Kern 
			$\begin{cases}
				Protonen:  & \text{- tragen je eine positive Elementarladung \, (+e)}\\ & \text{- Masse: $1.7 \cdot 10^{-27}$kg, \, gegen 2000x schwerer als 					Elektronen}\\
				Neutronen: & \text{- elektrisch neutral}\\ & \text{- Masse ungefähr wie Proton}
			\end{cases}$\\
			\item Hülle Elektronen
			\begin{itemize}
				\item[-] tragen je eine negative Elementarladungen (-e)\\
				\item[-] kleine Masse ($9.1 \cdot 10^{-31}kg$)\\
				\item[-] umkreisen den Kern auf Kreis- oder Ellipsen-bahnen\\
			\end{itemize}
		\end{itemize}
	\item Die elektrische Ladung ist stets an Materie gebunden (Protonen und Elektronen als Träger
der Elementarladungen).
	\item Masseinheit der Ladung Q: $ \lbrack Q \rbrack = 1  \;Coulomb = 1 C = 1 As$, $ e = 1.602\cdot 10^{-19} C$, $1 \; C = 6.241\cdot 10^{18}$ Elektronen
	\item Besitzt ein Atom oder eine Atomgruppe (Molekül) gleich viele Elektronen wie Protonen, so wirkt es nach aussen elektrisch neutral.Ionen sind Atome oder Moleküle, deren Ladungsneutralität wegen zu vielen oder zu wenigen Elektronen gestört ist:\\
				Aufnahme von Elektronen $\rightarrow$ neg. Ladung überwiegt $\rightarrow$ Anion\\
				Abgabe von Elektronen $\rightarrow$ pos. Ladung überwiegt $\rightarrow$ Kation\\
	\item Gleichnamige Ladungen stossen sich ab, ungleichnamige ziehen sich an $\rightarrow$ Coulombsche Kräfte 
	\item Jede elektrische Ladung erzeugt ein elektrisches Feld (und - falls sie sich bewegt - ein magnetisches Feld).\\
%				Ein elektrisches Feld ist ein besonderer Zustand des Raumes, welcher dadurch gekennzeichnet ist, dass auf elektrische Ladungen Kräfte wirken.
\end{itemize}

\subsection{Der elektrische Strom}
Bewegen sich gleichnamige Ladungsträger (meistens Elektronen), aufgrund einer Spannung oder eines elektr. Feldes, in eine bestimmte Richtung, so spricht man von einem elektrischen Strom.
$\rightarrow$ \textbf{Elektrischer Strom ist bewegte Ladung!}\\
Die Stromrichtung ist die Bewegungsrichtung der positiven Ladungsträger, d.h. die Gegenrichtung zur Bewegung der negativen Ladungsträger (Elektronen). Ein elektr. Stromkreis muss geschlossen sein damit Ladung fliessen kann, d.h. Leiter müssen als Verbindung verwendet werden. Strom entsteht nur, wenn sich Ladungen in einem Medium bewegen können $\rightarrow$ Unterscheidung zwischen Isolatoren (keine freie Ladungsträger, z.B. PVC, Porzellan), Halbleiter (freie Elektronen \& "Löcher", z.B. Halbleiter, Vakuum) und Leiter (freie Elektronen, Ionen, z.B. Metalle, Elektrolyte, Gase). Wirkung von Strom: Wärmewirkung, Magnetische Wirkung, Chemische Wirkung, Lichtwirkung, Physiologische Wirkung (z.B. Herzschrittmacher) \\ Messung des Stroms mittels Amperemeter\\
\begin{multicols}{2}
	\subsubsection{Stromstärke I}
	$ I=\frac{\Delta Q}{\Delta t}$ \hspace{10pt}  $\lbrack I \rbrack = A $\\
	Die Stromstärke besagt wie viel Ladungsmenge Q in einer bestimmten Zeit eine bestimmte Fläche durchfliesst. 
	\subsubsection{Stromdichte J}
	$ J=\frac {I}{A}$  \hspace{10pt}  $\lbrack J \rbrack = \frac {A}{m^2} $\\
	Eine kleinere Fläche A $\downarrow$ erzeugt eine höhere Stromdichte und grössere Wärmeerzeugung im Leiter
\end{multicols}
\subsubsection{Geschwindigkeit der freien Elektronen}
\begin{itemize}
	\item Ungeordnete thermische Bewegung, ähnlich wie bei Molekülen eines Gases: ca. 100 km/s (bei Raumtemperatur)\\
	\item Driftgeschwindigkeit = mittlere Fortbewegungsgeschwindigkeit der Elektronen in  (Gegen-)Stromrichtung: max. einige mm/s.\\
	\item Impulsgeschwindigkeit = Geschwindigkeit des Signals, das die Elektronen zum Driften veranlasst: Lichtgeschwindigkeit $\approx$ 300000 km/s\\
\end{itemize}

\subsection{Die elektrische Spannung}
Wenn irgendwo eine elektrische Spannung vorhanden ist, existiert immer auch ein elektrisches Feld. Auf Ladungen im elektrischen Feld wirkt die Coulom'sche Kraft F, welche bewirkt, dass die Ladungen sich in eine bestimmte Richtung bewegen und somit der Strom fliesst. Achtung: Falls die Kraft $\overrightarrow{F}$ nicht in dieselbe Richtung wie das Wegstück $l$ zeigt, ist der Kosinus des Zwischenwinkels $\alpha$ zu berücksichtigen! \\
	Spannung zw. A und B: $U _{AB} = \frac{W _{AB}}{Q}= E \cdot \overline{AB} = E \cdot l_{AB}$ \;
	$ \lbrack U \rbrack = \frac{\lbrack W \rbrack}{\lbrack Q \rbrack} = \frac{1J}{1C}=1V$	\\
	Elektrische Feldstärke: $\overrightarrow{E} = \frac{\overrightarrow{F}}{Q} = \overrightarrow{F} \cdot l \cdot \cos(\alpha) = \frac{V}{m}$\\
	Arbeit zw. A und B: $W_{AB} = \overrightarrow{F} \cdot l\textsubscript{AB} = Q \cdot E \cdot l \textsubscript{AB}$

\subsubsection{Das elektrische Potential $ \varphi $}
\underline{Definition}: Eine Spannung gegenüber einem (gemeinsamen) Bezugspunkt bezeichnet man auch als elektrisches Potential $\varphi$.\\
Dem Bezugspunkt wird meistens (aber nicht zwangsläufig) das Potential $\varphi = 0$ zugeordnet.\\
Die Spannung $U_{AB}$ zwischen zwei Punkten A und B (mit den Potentialen $\varphi _{A} $ bzw. $\varphi _{B}$) erhält man als Potentialdifferenz:\\ $ U_{AB}= \varphi _{A} - \varphi _{B}$\\

\subsection{Leistung und Energie}
$ P = \frac{\Delta W}{\Delta t}= I \cdot U = \frac{\Delta Q \cdot U}{\Delta t}$ \hspace{10pt} $\lbrack P \rbrack = W (Watt)$\\
$ \Delta W = e \cdot U_{AK} = \frac{1}{2} \cdot m_{e} \cdot v^2 = U \cdot I \cdot \Delta t$\hspace{10pt}$\lbrack W \rbrack = Ws = 1 Nm = 1J (Joul)$\\
