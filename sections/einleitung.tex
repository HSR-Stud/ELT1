\section{Einleitung}
\subsection{Die elektrische Ladung}
Alle physikalischen Erscheinungen, die wir als elektrisch bezeichnen, haben ihre Ursache in
ruhenden oder bewegten elektrischen Ladungen.
\linebreak

\begin{itemize}
	\item Ladungen im Atommodell von Bohr-Sommerfeld\\
	\begin{itemize}
			\item Kern 
			$\begin{cases}
				Protonen:  & \text{- tragen je eine positive Elementarladung(+e)}\\ & \text{- Masse: $1.7 \cdot 10^{-27}$kg,gegen 2000x schwerer als 					Elektronen}\\
				Neutronen: & \text{- elektrisch neutral}\\ & \text{- Masse ungef�hr wie Proton}
			\end{cases}$\\
			\item H�lle Elektronen
			\begin{itemize}
				\item[-] tragen je eine negative Elementarladungen (-e)\\
				\item[-] kleine Masse ($9.1 \cdot 10^{-31}kg$)\\
				\item[-] umkreisen den Kern aus Kreis- oder Ellipsenbahnen\\
			\end{itemize}
		\end{itemize}
	\item Kleinstm�gliche Ladung: Elementarladung e $\rightarrow$ jede elektr.Ladungen ist ein ganz Zahliges Vielfaches von e\\
	\item Die elektrische Ladung ist stets an Materie gebunden (Protonen und Elektronen als Tr�ger
der Elementarladungen besitzen eine Masse).
	\item Masseinheit der Ladung Q: $ \lbrack Q \rbrack = 1 Coulomb = 1 C = 1 As$, $ e = 1.602\cdot 10^{-19} C$, $1 C = 6.25\cdot 10^{18}$ Elektronen
	\item Besitzt ein Atom oder eine Atomgruppe (Molek�l) gleich viele Elektronen wie Protonen, so wirkt es nach aussen elektrisch neutral.\\
				Ionen sind Atome oder Molek�le, deren Ladungsneutralit�t wegen zu vielen oder zu wenigen Elektronen gest�rt ist:\\
				Aufnahme von Elektronen $\rightarrow$ neg. Ladung �berwiegt $\rightarrow$ Anion\\
				Abgabe von Elektronen $\rightarrow$ pos. Ladung �berwiegt $\rightarrow$ Kation\\
	\item Gleichnamige Ladungen stossen sich ab, ungleichnamige ziehen sich an $\rightarrow$ Coulombsche Kr�fte 
	\item Jede elektrische Ladung erzeugt ein elektrisches Feld (und - falls sie sich bewegt - ein magnetisches Feld).\\
				Ein elektrisches Feld ist ein besonderer Zustand des Raumes, welcher dadurch gekennzeichnet ist, dass auf elektrische Ladungen Kr�fte wirken.
\end{itemize}

\subsection{Der elektrische Strom}
Bewegen sich gleichnamige Ladungstr�ger - meistens Elektronen - in einer bestimmten Richtung, so spricht man von einem elektrischen Strom.
$\rightarrow$ \underline{Elektrischer Strom ist bewegte Ladung!}\\
\begin{multicols}{2}
	\subsubsection{Stromst�rke I}
	$ I=\frac{\Delta Q}{\Delta t}$ \hspace{10pt}  $\lbrack I \rbrack = A $\\
	\subsubsection{Stromdichte J}
	$ J=\frac {I}{A}$  \hspace{10pt}  $\lbrack J \rbrack = \frac {A}{m^2} $\\
\end{multicols}
\subsubsection{Geschwindigkeit der freien Elektronen}
\begin{itemize}
	\item Ungeordnete thermische Bewegung, �hnlich wie bei Molek�len eines Gases:ca. 100 km/s (bei Raumtemperatur)\\
	\item Driftgeschwindigkeit = mittlere Fortbewegungsgeschwindigkeit der Elektronen in  (Gegen-)Stromrichtung: max. einige mm/s.\\
	\item Impulsgeschwindigkeit = Geschwindigkeit des Signals, das die Elektronen zur Drift veranlasst: Lichtgeschwindigkeit $\approx$ 300000 km/s\\
\end{itemize}
\newpage
\subsection{Die elektrische Spannung}
\begin{multicols}{2}
	$ E = \frac{F}{Q}$ \hspace{5pt}= Feldst�rke $ \lbrack E \rbrack = \frac{V}{m}$,\hspace{10pt} 
	$ U _{AB} = \frac{W _{AB}}{Q}= E \cdot \overline{AB}$ $ \lbrack U \rbrack = \frac{\lbrack W \rbrack}{\lbrack Q \rbrack} = \frac{1J}{1C}=1V$\\
\end{multicols}
\subsubsection{Das elektrische Potential $ \varphi $}
\underline{Definition}: Eine Spannung gegen�ber einem (gemeinsamen) Bezugspunkt bezeichnetman auch als elektrisches Potential $\varphi$.\\
Dem Bezugspunkt wird meistens (aber nicht zwangsl�ufig) das Potential $\varphi = 0$ zugeordnet.\\
Die Spannung $U_{AB}$ zwischen zwei Punkten A und B (mit den Potentialen $\varphi _{A} $ bzw. $\varphi _{B}$) erh�lt man als Potentialdifferenz:\\ $ U_{AB}= \varphi _{A} - \varphi _{B}$\\

\subsection{Leistung und Energie}
$ P = \frac{\Delta W}{\Delta t}= I \cdot U = \frac{\Delta Q \cdot U}{\Delta t}$ \hspace{10pt} $\lbrack P \rbrack = W (Watt)$\\
$ \Delta W = e \cdot U_{AK} = \frac{1}{2} \cdot m_{e} \cdot v^2 = U \cdot I \cdot \Delta t$\hspace{10pt}$\lbrack W \rbrack = Ws = 1 Nm = 1J (Joul)$\\
